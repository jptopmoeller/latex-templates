
% Basics und Codierung
% ===========================================================
  \usepackage{wwustyle}
  \usepackage[ngerman]{babel}
  \usepackage[T1]{fontenc}
  \usepackage[german=quotes]{csquotes}
  \usepackage{etex}
% ===========================================================


% Fonts und Typographie
% ===========================================================
  \usepackage{sourcecodepro}
  \usepackage[default]{sourcesanspro}
  \usepackage{nimbusmononarrow}
  %\usepackage{ellipsis}
  \newcommand{\bet}[1]{\textbf{\color{maincolor}#1}}
  \newcommand{\minor}[1]{\textcolor{black!50}{#1}}
  \newcommand{\minoritem}{\item[{\textcolor{black!50}{$\blacktriangleright$}}]}
  \newcommand{\code}[1]{\texttt{#1}}
  \usepackage{xspace}
  \makeatletter 
  \xspaceaddexceptions{\grqq \grq \csq@qclose@i \} } 
  \makeatother
  \usefonttheme[onlymath]{serif}
  \usepackage{multicol}
  \newcommand{\hyper}[1]{\bet{\underline{\smash{#1}}}}
% ===========================================================

% Farben  
% ===========================================================
  \usepackage{xcolor}
  \definecolor{fbblau}{HTML}{3078AB}
  \definecolor{mediumgray}{gray}{.65}
  \definecolor{blackberry}{rgb}{0.53, 0.0, 0.25}
% ===========================================================

  
% Mathe-Pakete
% ===========================================================
  \usepackage{mathtools}
  \usepackage{amssymb}
  \usepackage[bigdelims]{newtxmath}
  \usepackage{wasysym}
  
  % ggfs. anpassen, um Abstände nach Formeln zu modifizieren
  \newcommand{\zerodisplayskips}{%
    %\setlength{\abovedisplayskip}{0pt}%
    %\setlength{\belowdisplayskip}{0pt}%
    %\setlength{\abovedisplayshortskip}{0pt}%
    %\setlength{\belowdisplayshortskip}{0pt}
    %\setlength{\multicolsep}{0pt}
  }
  \appto{\normalsize}{\zerodisplayskips}
  \appto{\small}{\zerodisplayskips}
  \appto{\footnotesize}{\zerodisplayskips}
% ===========================================================

% TikZ
% ===========================================================
  \usepackage{tikz}
  \usepackage{tikz-cd}          % kommutative Diagramme
  \usetikzlibrary{arrows.meta}      % mehr Pfeile!
  \usetikzlibrary{calc}
  \tikzset{>=Latex}            % Standard-Pfeilspitze
% ===========================================================

% listings
% ===========================================================
  \usepackage{listingsutf8}
  \lstset{
    belowcaptionskip=1\baselineskip,
    breaklines=true,
    showstringspaces=false,
    basicstyle=\ttfamily,
    keywordstyle=\bfseries\color{green!40!black},
    commentstyle=\itshape\color{purple!40!black},
    stringstyle=\color{orange},
    numbers=left,
    numberstyle=\footnotesize\ttfamily\color{maincolor},
    inputencoding=utf8/latin1,
    tabsize=4,
  }
% ===========================================================

% Bibliographie
% ===========================================================
  \usepackage[%
    backend=biber,
    sortlocale=auto,
    natbib,
    hyperref,
    backref=false,
    style=numeric,
  ]{biblatex}
  \addbibresource{bibliography.bib}
  \setbeamertemplate{bibliography item}[text]
  \renewcommand*{\bibfont}{\footnotesize}
  \hypersetup{citecolor=maincolor}
  \nocite{*}    % auch Quellen ohne Nennungen ausgeben
% ===========================================================

% beamer-Konfiguration
% ===========================================================
  \setbeamertemplate{section in toc}[sections numbered]  % nummeriert sections in Inhaltsverzeichnis
  
  % Inhaltsverzeichnis bei section-Wechsel anzeigen
  \AtBeginSection[]{
    \begin{frame}[t]
      \tableofcontents[currentsection, hidesubsections, hideothersubsections,sectionstyle=show/shaded]
    \end{frame}}
% ===========================================================


%%%%%%%%%%%%%%%%%%%%%%%%%%%%%%%%%%%%%%%%%%%%%%%%%%%%%%%%%%%
%%% Ab hier folgen nur noch vordefinierte Shortcuts %%%
%%%%%%%%%%%%%%%%%%%%%%%%%%%%%%%%%%%%%%%%%%%%%%%%%%%%%%%%%%%

\newcommand{\BB}{\mathbb{B}}
\newcommand{\CC}{\mathbb{C}}
\newcommand{\NN}{\mathbb{N}}
\newcommand{\QQ}{\mathbb{Q}}
\newcommand{\RR}{\mathbb{R}}
\newcommand{\ZZ}{\mathbb{Z}}
\newcommand{\oh}{\mathcal{O}}

\newcommand{\ol}[1]{\overline{#1}}
\newcommand{\wt}[1]{\widetilde{#1}}
\newcommand{\wh}[1]{\widehat{#1}}

\DeclareMathOperator{\id}{id}         % Identität
\DeclareMathOperator{\pot}{\mathcal{P}}    % Potenzmenge

% Klammerungen und ähnliches
\DeclarePairedDelimiter{\absolut}{\lvert}{\rvert}    % Betrag
\DeclarePairedDelimiter{\ceiling}{\lceil}{\rceil}    % aufrunden
\DeclarePairedDelimiter{\Floor}{\lfloor}{\rfloor}    % aufrunden
\DeclarePairedDelimiter{\Norm}{\lVert}{\rVert}      % Norm
\DeclarePairedDelimiter{\sprod}{\langle}{\rangle}    % spitze Klammern
\DeclarePairedDelimiter{\enbrace}{(}{)}          % runde Klammern
\DeclarePairedDelimiter{\benbrace}{\lbrack}{\rbrack}  % eckige Klammern
\DeclarePairedDelimiter{\penbrace}{\{}{\}}        % geschweifte Klammern
\newcommand{\Underbrace}[2]{{\underbrace{#1}_{#2}}}   % bessere Unterklammerungen
% Kurzschreibweisen für Faule und Code-Vervollständigung
\newcommand{\abs}[1]{\absolut*{#1}}
\newcommand{\ceil}[1]{\ceiling*{#1}}
\newcommand{\flo}[1]{\Floor*{#1}}
\newcommand{\no}[1]{\Norm*{#1}}
\newcommand{\sk}[1]{\sprod*{#1}}
\newcommand{\enb}[1]{\enbrace*{#1}}
\newcommand{\penb}[1]{\penbrace*{#1}}
\newcommand{\benb}[1]{\benbrace*{#1}}
\newcommand{\stack}[2]{\makebox[1cm][c]{$\stackrel{#1}{#2}$}}