%!TEX TS-program = pdflatex
%!TEX TS-options = -shell-escape
% Author: Phil Steinhorst, p.st@wwu.de
% https://gitlab.com/phist91/latex-templates

\newcommand{\obenlinks}{Übungen zur Vorlesung Informatik I}   % hier Name der Veranstaltung eintragen

% Basics und Codierung
% ===========================================================
  \usepackage{wwustyle}
  \usepackage[ngerman]{babel}
  \usepackage[T1]{fontenc}
  \usepackage[german=quotes]{csquotes}
  \usepackage{etex}
% ===========================================================


% Fonts und Typographie
% ===========================================================
  \usepackage{sourcecodepro}
  \usepackage[default]{sourcesanspro}
  \usepackage{nimbusmononarrow}
  %\usepackage{ellipsis}
  \newcommand{\bet}[1]{\textbf{\color{maincolor}#1}}
  \newcommand{\minor}[1]{\textcolor{black!50}{#1}}
  \newcommand{\minoritem}{\item[{\textcolor{black!50}{$\blacktriangleright$}}]}
  \newcommand{\code}[1]{\texttt{#1}}
  \usepackage{xspace}
  \makeatletter 
  \xspaceaddexceptions{\grqq \grq \csq@qclose@i \} } 
  \makeatother
  \usefonttheme[onlymath]{serif}
  \usepackage{multicol}
  \newcommand{\hyper}[1]{\bet{\underline{\smash{#1}}}}
% ===========================================================

% Farben  
% ===========================================================
  \usepackage{xcolor}
  \definecolor{fbblau}{HTML}{3078AB}
  \definecolor{mediumgray}{gray}{.65}
  \definecolor{blackberry}{rgb}{0.53, 0.0, 0.25}
% ===========================================================

  
% Mathe-Pakete
% ===========================================================
  \usepackage{mathtools}
  \usepackage{amssymb}
  \usepackage[bigdelims]{newtxmath}
  \usepackage{wasysym}
  
  % ggfs. anpassen, um Abstände nach Formeln zu modifizieren
  \newcommand{\zerodisplayskips}{%
    %\setlength{\abovedisplayskip}{0pt}%
    %\setlength{\belowdisplayskip}{0pt}%
    %\setlength{\abovedisplayshortskip}{0pt}%
    %\setlength{\belowdisplayshortskip}{0pt}
    %\setlength{\multicolsep}{0pt}
  }
  \appto{\normalsize}{\zerodisplayskips}
  \appto{\small}{\zerodisplayskips}
  \appto{\footnotesize}{\zerodisplayskips}
% ===========================================================

% TikZ
% ===========================================================
  \usepackage{tikz}
  \usepackage{tikz-cd}          % kommutative Diagramme
  \usetikzlibrary{arrows.meta}      % mehr Pfeile!
  \usetikzlibrary{calc}
  \tikzset{>=Latex}            % Standard-Pfeilspitze
% ===========================================================

% listings
% ===========================================================
  \usepackage{listingsutf8}
  \lstset{
    belowcaptionskip=1\baselineskip,
    breaklines=true,
    showstringspaces=false,
    basicstyle=\ttfamily,
    keywordstyle=\bfseries\color{green!40!black},
    commentstyle=\itshape\color{purple!40!black},
    stringstyle=\color{orange},
    numbers=left,
    numberstyle=\footnotesize\ttfamily\color{maincolor},
    inputencoding=utf8/latin1,
    tabsize=4,
  }
% ===========================================================

% Bibliographie
% ===========================================================
  \usepackage[%
    backend=biber,
    sortlocale=auto,
    natbib,
    hyperref,
    backref=false,
    style=numeric,
  ]{biblatex}
  \addbibresource{bibliography.bib}
  \setbeamertemplate{bibliography item}[text]
  \renewcommand*{\bibfont}{\footnotesize}
  \hypersetup{citecolor=maincolor}
  \nocite{*}    % auch Quellen ohne Nennungen ausgeben
% ===========================================================

% beamer-Konfiguration
% ===========================================================
  \setbeamertemplate{section in toc}[sections numbered]  % nummeriert sections in Inhaltsverzeichnis
  
  % Inhaltsverzeichnis bei section-Wechsel anzeigen
  \AtBeginSection[]{
    \begin{frame}[t]
      \tableofcontents[currentsection, hidesubsections, hideothersubsections,sectionstyle=show/shaded]
    \end{frame}}
% ===========================================================


%%%%%%%%%%%%%%%%%%%%%%%%%%%%%%%%%%%%%%%%%%%%%%%%%%%%%%%%%%%
%%% Ab hier folgen nur noch vordefinierte Shortcuts %%%
%%%%%%%%%%%%%%%%%%%%%%%%%%%%%%%%%%%%%%%%%%%%%%%%%%%%%%%%%%%

\newcommand{\BB}{\mathbb{B}}
\newcommand{\CC}{\mathbb{C}}
\newcommand{\NN}{\mathbb{N}}
\newcommand{\QQ}{\mathbb{Q}}
\newcommand{\RR}{\mathbb{R}}
\newcommand{\ZZ}{\mathbb{Z}}
\newcommand{\oh}{\mathcal{O}}

\newcommand{\ol}[1]{\overline{#1}}
\newcommand{\wt}[1]{\widetilde{#1}}
\newcommand{\wh}[1]{\widehat{#1}}

\DeclareMathOperator{\id}{id}         % Identität
\DeclareMathOperator{\pot}{\mathcal{P}}    % Potenzmenge

% Klammerungen und ähnliches
\DeclarePairedDelimiter{\absolut}{\lvert}{\rvert}    % Betrag
\DeclarePairedDelimiter{\ceiling}{\lceil}{\rceil}    % aufrunden
\DeclarePairedDelimiter{\Floor}{\lfloor}{\rfloor}    % aufrunden
\DeclarePairedDelimiter{\Norm}{\lVert}{\rVert}      % Norm
\DeclarePairedDelimiter{\sprod}{\langle}{\rangle}    % spitze Klammern
\DeclarePairedDelimiter{\enbrace}{(}{)}          % runde Klammern
\DeclarePairedDelimiter{\benbrace}{\lbrack}{\rbrack}  % eckige Klammern
\DeclarePairedDelimiter{\penbrace}{\{}{\}}        % geschweifte Klammern
\newcommand{\Underbrace}[2]{{\underbrace{#1}_{#2}}}   % bessere Unterklammerungen
% Kurzschreibweisen für Faule und Code-Vervollständigung
\newcommand{\abs}[1]{\absolut*{#1}}
\newcommand{\ceil}[1]{\ceiling*{#1}}
\newcommand{\flo}[1]{\Floor*{#1}}
\newcommand{\no}[1]{\Norm*{#1}}
\newcommand{\sk}[1]{\sprod*{#1}}
\newcommand{\enb}[1]{\enbrace*{#1}}
\newcommand{\penb}[1]{\penbrace*{#1}}
\newcommand{\benb}[1]{\benbrace*{#1}}
\newcommand{\stack}[2]{\makebox[1cm][c]{$\stackrel{#1}{#2}$}}  % Präambel (ohne die geht nichts!)


\begin{document}
\begin{center} % Kopfbereich mit Infos zur Abgabegruppe
  \begin{tabular}{|rlp{4cm}rl|}
    \hline
    \textbf{Übungsblatt:} & 13 &  & \textbf{1. Abgabepartner*in:} & Ada Lovelace \\
    \textbf{Aufgabe:} & 37 &  & \textbf{2. Abgabepartner*in:} & Donald Knuth \\
    \textbf{Abgabegruppe:} & 42 &  & \textbf{3. Abgabepartner*in:} & Alan Turing  \\ \hline
  \end{tabular}
\end{center} 

Dies ist eine \LaTeX-Vorlage für Übungszettelabgaben im Fach Informatik, die das \texttt{minted}-Paket nutzt.
Im Gegensatz zu den standardmäßigen \texttt{listings}-Umgebungen, für die in meinem Git-Repository\footnote{\url{https://github.com/phist91/latex-templates}} ebenfalls eine Vorlage bereitgestellt wird, muss man einige Vorbereitungen treffen, damit \texttt{minted} (und auch diese Vorlage) benutzt werden kann.
Belohnt wird man dafür mit einem schöneren und deutlich differenzierteren automatischen Syntax Highlighting.
Nachfolgend gibt es ein paar Hinweise, wie diese Vorlage zu verwenden ist.

\subsection*{Disclaimer}
Ich kann leider grundsätzlich keinen \LaTeX-Support anbieten und verweise daher auf gängige Suchmaschinen und die \TeX-Community von StackExchange\footnote{\url{https://tex.stackexchange.org/}}.    
Sucht man den LaTeX-Befehl für ein bestimmtes Symbol, ist Detexify\footnote{\url{http://detexify.kirelabs.org/}} praktisch.

\subsection*{Verwendung}
\texttt{minted} basiert auf der Programmiersprache Python\footnote{\url{https://www.python.org/downloads/}}.
Damit man es zum Laufen bekommt, muss man allerdings keine Kenntnisse diesbezüglich mitbringen, sondern lediglich eine aktuelle Version installieren.
Im Anschluss muss noch \texttt{Pygments} eingerichtet werden.
Details dazu finden sich in Abschnitt 2.1 der \texttt{minted}-Dokumentation\footnote{\url{https://ctan.org/pkg/minted?lang=de}}.

Als Test sollte man versuchen, die Datei \texttt{abgaben-minted.tex} in einem TeX-Editor zu öffnen und zu kompilieren.
Dabei handelt es sich um den TeX-Code zu genau diesem Dokument.
Wenn das nicht klappt, kann das folgende Gründe haben:
\begin{itemize}
  \item In den Umgebungsvariablen des Systems fehlt das Verzeichnis der \LaTeX-Distribution oder zu Python.
  \item Der Compiler wird nicht mit der Shell-Escape-Option aufgerufen.
    Ggfs. sollte in den Einstellungen des Editors nachgeschaut werden, dass \texttt{pdflatex} mit der Option \texttt{-shell-escape} aufgerufen wird.
  \item Irgendwas anderes. Man prüfe die Ausgabe des Compilers und ziehe ggfs. die \texttt{minted}-Dokumentation zurate.
\end{itemize}
Wenn alles geklappt hat, kann man diese Vorlage verwenden.
Ich habe als Einstiegshilfe unten verschiedene Beispiele -- unter anderem zum Einbinden von Quellcode -- angehängt.
Die Verwendung der entsprechenden TeX-Befehle sollte anhand der Datei \texttt{abgaben-minted.tex} erschließbar sein.
Weitere Hinweise und Konfigurationsmöglichkeiten findet man in der Paketdokumentation.

\textbf{Nicht vergessen:} Ein automatisches Syntax Highlighting entbindet \textbf{niemals} von der Pflicht, gut lesbaren und gut nachvollziehbaren Code zu produzieren.

\begin{flushright}
Phil Steinhorst \\
\url{https://github.com/phist91/latex-templates}
\end{flushright}

\newpage

%%%%%%%%%%%%%%%%%%%%%%%%%%%%%%%%%%%%%%%%%%%%%%%%%%%%%%%%%%%%%%%
\section*{Beispiele}
\subsection*{Textformatierungen}
Eine Auflistung:
\begin{itemize}
  \item Dieser Text ist \textbf{fett}.
  \item Dieser Text ist \textit{kursiv}.
  \item Dieser Text ist \underline{unterstrichen}.
  \item Dieser Text ist \textcolor{red}{rot}.
  \item Dieser Text ist \href{https://gitlab.com/phist91/latex-templates}{anklickbar.}
\end{itemize}

Eine Aufzählung:
\begin{enumerate}[(a)]
  \item Eine Liste,
  \item aber mit
  \item Nummerierung!
\end{enumerate}

Eine zentrierte Tabelle:
\begin{center}
  \begin{tabular}{|c|c|} \hline
    Etwas         &  gewöhnungsbedürftig,         \\ \hline
    \multicolumn{2}{|c|}{aber vielseitig und machbar!} \\ \hline
  \end{tabular}
\end{center}

Die gleiche Tabelle, aber mit Beschreibung:
\begin{table}[h]
  \begin{center}
    \begin{tabular}{|c|c|} \hline
      Etwas         &  gewöhnungsbedürftig,        \\ \hline
      \multicolumn{2}{|c|}{aber vielseitig und machbar!} \\ \hline
    \end{tabular}
  \end{center}
  \caption{Viele Editoren bieten Assistenten zum Anlegen von Tabellen wie dieser.}
\end{table}

%%%%%%%%%%%%%%%%%%%%%%%%%%%%%%%%%%%%%%%%%%%%%%%%%%%%%%%%%%%%%%%
\subsection*{Grafiken}
Eine eingebundene Grafikdatei:
\begin{center}
  \includegraphics[keepaspectratio,width=1cm]{lemon.png}
\end{center}
\newpage
Die gleiche Grafik, in verschiedenen Größen:
\begin{figure}[h]
  \begin{center}
    \subcaptionbox{klein}{\includegraphics[keepaspectratio,width=1.5cm]{lemon.png}}
    \subcaptionbox{mittel}{\includegraphics[keepaspectratio,width=2cm]{lemon.png}}
    \subcaptionbox{groß}{\includegraphics[keepaspectratio,width=3cm]{lemon.png}}
    \caption{Zitronen in verschiedenen Größen.}
  \end{center}
\end{figure}

Eine TikZ-Grafik mit Beschreibung:
\begin{figure}[h]
  \centering
  \begin{tikzpicture}[->, auto, node distance= 1.5cm and 3cm, on grid, semithick,
    state/.style={circle, draw, font=\ttfamily, minimum width= 0.7cm}]
    \node[state] (A) at (0,0) {a};
    \node[state,gray] (B) at (1.5,1) {b};
    \node[state] (C) at (3,1) {c};
    \node[state] (D) at (2.5,-0.5) {d};
    \node[state] (E) at (4,-0.5) {e};
    
    \draw[->] (-0.5,1) node[above]{\dots} -- (A);
    \draw[->,gray] (B) -- (C);
    \draw[->,gray] (B) -- (D);
    \draw[->] (C) to[bend angle=20,bend right] (E);
    \draw[->] (E) to[bend angle=20,bend right] (C);
    \draw[->] (D) -- (C);
    \draw[->] (E) -- (4.5,0.5) node[above]{\dots};
    \draw[->] (A) to[bend angle=20,bend right] (D);
    
    \draw (A) node[below=0.35cm,RoyalBlue2,font=\bfseries]{1};
    \draw (B) node[below=0.35cm,RoyalBlue2,font=\bfseries]{0};
    \draw (C) node[right=0.35cm,RoyalBlue2,font=\bfseries]{2};
    \draw (D) node[below=0.35cm,RoyalBlue2,font=\bfseries]{1};
    \draw (E) node[below=0.35cm,RoyalBlue2,font=\bfseries]{1};
    \path (E) edge [loop right] (E);
  \end{tikzpicture}
  \caption{TikZ ist ein sehr mächtiges Paket zur Erstellung ansprechender Vektorgrafiken.}
\end{figure}

%%%%%%%%%%%%%%%%%%%%%%%%%%%%%%%%%%%%%%%%%%%%%%%%%%%%%%%%%%%%%%%
\subsection*{Mathematische Formeln}

Die Formel $a^2 + b^2 = c^2$ ist kompakt und passt in den Fließtext.

Die Formel $x := \frac{1}{1+\frac{1}{1+\frac{1}{2}}}$ sprengt hingegen die Zeilenhöhe, was innerhalb eines größeren Absatzes sehr unschön aussehen kann, wie man hier sieht.

Besser: Einzeilige Formeln absetzen und zentrieren:
\[
x := \frac{1}{1+\frac{1}{1+\frac{1}{2}}}
\]

Und mehrzeilige Formeln? Bitte nicht so:

$f(x) = x^2 + 2x - 7$ \\
$f'(x) = 2x+2$ \\
$f''(x) = 2$

Viel besser: Abgesetzt und ausgerichtet:
\begin{align*}
  f(x) &= x^2 + 2x - 7 \\
  f'(x) &= 2x+2 \\
  f''(x) &= 2
\end{align*}

Formeln können auch nummeriert werden:
\begin{align}
  \sum_{i=0}^{n} x^i &= \frac{1-x^{n+1}}{1-x}       \label{formel-A} \\
  \sum_{i=0}^{n} \frac{1}{2^i} &= 2 - \frac{1}{2^n} \label{formel-B}
\end{align}
\dots um sie später zu referenzieren: Formel \eqref{formel-B} ist ein Spezialfall von Formel \eqref{formel-A}.

Eine Funktion:
\begin{align*}
  f \colon \mathbb{R} &\longrightarrow \mathbb{R} \\
  x &\longmapsto x^2
\end{align*}

Geht auch mit Fallunterscheidung:
\begin{align*}
  \operatorname{sgn} \colon \mathbb{R} &\longrightarrow \mathbb{R} \\
  x &\longmapsto \begin{cases}
    1, & \text{falls } x > 0 \\
    0, & \text{falls } x = 0 \\
    -1, & \text{falls } x < 0
  \end{cases}
\end{align*}

Eine Matrix:
\[
A := \begin{pmatrix}
  1 & a & a^2 & a^3 \\
  1 & b & b^2 & b^3 \\
  1 & c & c^2 & c^3 \\
  1 & d & d^2 & d^3
\end{pmatrix}
\]

\textbf{Wichtig!} Niemals Fließtext ohne Befehle wie \texttt{\textbackslash text\{...\}} in Formel-Umgebungen setzen:
\[
x_{Staffel}
\]
Das führt zu Problemen beim Zeichenabstand und sieht komisch aus.

%%%%%%%%%%%%%%%%%%%%%%%%%%%%%%%%%%%%%%%%%%%%%%%%%%%%%%%%%%%%%%%

\subsection*{Beispiel für direkt eingegebenen Java-Code:}
\begin{minted}{java}
import java.util.*;
public class Hallo{
  public static void main( String[] args ) {
    System.out.print("Hallo Welt!");
  }
}
\end{minted}

%%%%%%%%%%%%%%%%%%%%%%%%%%%%%%%%%%%%%%%%%%%%%%%%%%%%%%%%%%%%%%%

\subsection*{Beispiel für direkt eingegebenen C-Code:}
\begin{minted}{c}
#include <stdio.h>
int main(int argc, char** argv){
  int i;
  for(i = 0; i < argc; i++){
    printf("%s \n", argv[i]);
  }
  return 0;
}
\end{minted}

%%%%%%%%%%%%%%%%%%%%%%%%%%%%%%%%%%%%%%%%%%%%%%%%%%%%%%%%%%%%%%%
\subsection*{Beispiel für Java-Code, der aus einer eigenen Datei eingebunden wird:}

\inputminted{java}{programm.java}

%%%%%%%%%%%%%%%%%%%%%%%%%%%%%%%%%%%%%%%%%%%%%%%%%%%%%%%%%%%%%%%
\subsection*{Beispiel für C-Code, der aus einer eigenen Datei eingebunden wird:}

\inputminted{c}{programm.c}


%%%%%%%%%%%%%%%%%%%%%%%%%%%%%%%%%%%%%%%%%%%%%%%%%%%%%%%%%%%%%%%

\textbf{Beispiel für Inline-Java-Code:} \mintinline{c}{this.toScan = new LinkedList<>();}

\textbf{Beispiel für Inline-C-Code:} \mintinline{c}{int main(int argc, char** argv)}

%%%%%%%%%%%%%%%%%%%%%%%%%%%%%%%%%%%%%%%%%%%%%%%%%%%%%%%%%%%%%%%


\end{document}